% !TEX root = ../main.tex

\chapter{Definitions}

\section{Zero-Knowledge Proofs}
\label{app:zkp}

\begin{definition}[Special Soundness]
For a $\Sigma$-protocol, if the witness $w$ can be extracted from any two accepting conversations on the same input $x$ with the same message but different challenge, we call this special soundness of $\Sigma$-protocol. Particularly, special soundness implies soundness.
\end{definition}

\begin{definition}[Knowledge Soundness in the Algebraic Group Model]

\end{definition}

\begin{definition}[Honest Verifier Zero Knowledge]
Honest verifier zero knowledge, or HVZK, is for a $\Sigma$-protocol, if there exists a p.p.t simulator $\mathcal{S}$ such that the transcript produced by $\mathcal{S}$ has the same distribution as the transcript of a conversation between the honest $\prv$ and $\vrf$ on the same input.
\end{definition}

\begin{definition}[Special Honest Verifier Zero Knowledge]
Special honset verifier zero knowledge, or special HVZK, is for a $\Sigma$-protocol, if there exists a p.p.t simulator such that additionally given the challenge $c$ to $\mathcal{S}$, the transcript produced by $\mathcal{S}$ has the same distribution as the transcript of a conversation between the honest $\prv$ and $\vrf$ on the same input, even if for the case that the witness for the statement does not exist.
\end{definition}

\section{Proof of Solvency}
\label{app:defs}

Definitions~\ref{def:1} and~\ref{def:2} are taken largely verbatim from the Provisions paper at CCS 2015~\cite{provisions}. Let $\mathcal{A}$ (exchange-controlled addresses) and $\mathcal{A}'$ (anonymity set of addresses) denote mappings $(y = g^x) \mapsto \text{bal}(y)$ where $\mathcal{A} \subseteq \mathcal{A}'$, $y$ is the public key corresponding to a Bitcoin address with private key $x$ and $\text{bal}(y)$ is the amount of currency, or assets, observably spendable by this key on the blockchain. Let $\mathcal{L}$ denote a mapping $\text{ID} \mapsto \ell$ where $\ell$ is the amount of currency, or liabilities, owed by the exchange to each user identified by the unique identity ID. A balance is a positive integer in $[0,\mathsf{MaxETH}]$ for a known upper-bound $\mathsf{MaxETH}$. The size of $\mathcal{A}'$ is known, the size of $\mathcal{A}$ is generally unknown (beyond being less than or equal to $\mathcal{A}'$), and the size of $\mathcal{L}$ is generally unknown (see Definition \ref{def:2}(4) below). 

\begin{definition}[Valid Pair]
\label{def:1}

We say that $\mathcal{A}$ and $\mathcal{L}$ are a valid pair with respect to a positive integer $\mathsf{MaxETH}$ iff $\forall \text{ID} \in \mathcal{L}$,

\begin{itemize}
\item $\sum_{y \in \mathcal{A}} \mathcal{A}[y] - \sum_{\mathsf{ID} \in \mathcal{L}} \mathcal{L}[\text{ID}] \geq 0 \quad$
\item $0 \leq \mathcal{L}[\text{ID}] \leq \mathsf{MaxETH}$
\end{itemize}


Consider an interactive protocol ProveSolvency run between an exchange $\mathcal{E}$ and user $\mathcal{U}$ such that

\begin{itemize}
\item $\mathsf{output}_{\mathcal{E}}^{\mathsf{ProveSolvency}}(1^k, \mathsf{MaxETH}, \mathcal{A}, \mathcal{L}, \mathcal{A}') = \emptyset$
\item $\mathsf{output}_{\mathcal{U}}^{\mathsf{ProveSolvency}}(1^k, \mathsf{MaxETH}, \mathcal{A}', \mathsf{ID}, \ell) \in \{\text{ACCEPT}, \text{REJECT}\}$
\end{itemize}
   
\end{definition}

For brevity, we refer to these as $\text{out}_{\mathcal{E}}$ and $\text{out}_{\mathcal{U}}$ respectively. Next we define, with reference to the valid pair definition, a privacy-preserving proof of solvency.


\begin{definition}[Privacy-Preserving Proof of Solvency]
\label{def:2}

A privacy-preserving proof of solvency is a probabilistic polynomial-time interactive protocol ProveSolvency, with inputs/outputs as above, such that the following properties hold:

\begin{enumerate}
\item \textit{Correctness}. If $\mathcal{A}$ and $\mathcal{L}$ are a valid pair and $\mathcal{L}[\text{ID}] = \ell$, then $\Pr[\text{out}_{\mathcal{U}} = \text{ACCEPT}] = 1$.
\item \textit{k-Soundness}. If $\mathcal{A}$ and $\mathcal{L}$ are instead not a valid pair, or if $\mathcal{L}[\text{ID}] \neq \ell$, then $\Pr[\text{out}_{\mathcal{U}} = \text{REJECT}] \geq 1 - \mathsf{negl}(k)$.
\item \textit{Ownership}. For all valid pairs $\mathcal{A}$ and $\mathcal{L}$, if $\Pr[\text{out}_{\mathcal{U}} = \text{ACCEPT}] = 1$, then the exchange must have ‘known’ the private keys associated with the public keys in $\mathcal{A}$; i.e., there exists an extractor that, given $\mathcal{A}$, $\mathcal{L}$, and rewindable black-box access to exchange $\mathcal{E}$, can produce $x$ for all $y \in \mathcal{A}$.
\item \textit{Privacy}. A potentially dishonest user $\mathcal{U}'$ interacting with an honest exchange $\mathcal{E}$ cannot learn anything about a valid pair $\mathcal{A}$ and $\mathcal{L}$ beyond its validity and $\mathcal{L}[\text{ID}]$ (and possibly $|\mathcal{A}|$ and $|\mathcal{L}|$); i.e., even a cheating user cannot distinguish between an interaction using the real pair $\mathcal{A}$ and $\mathcal{L}$ and any other (equally sized) valid pair $\hat{\mathcal{A}}$ and $\hat{\mathcal{L}}$ such that $\hat{\mathcal{L}}[\text{ID}] = \mathcal{L}[\text{ID}]$.
\end{enumerate}
\end{definition}
