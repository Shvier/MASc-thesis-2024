% !TEX root = ../main.tex

\chapter{Protocols}

\section{Range Proof for Single Value with $\kzgzk$}
\label{sec:rpzk}
Assume $\prv$ is given $x$ and $\prv$ computes $f(X)$ using the above range proof (\ref{sec:range}). $\prv$ wants to prove $f(X)$ satisfies the second and the third condition, \ie $w_2$ and $w_3$ are vanishing over $H$. They run the protocol as follows:
\begin{enumerate}
    \item $\prv$ \textit{generates two random numbers $\omega^{\prime},\omega^{\prime\prime}\in\mathbb{F}\setminus{H}$ and another two random numbers $\alpha,\beta\in\mathbb{F}$}
    \item $\prv$ \textit{interpolates $f$ at two more points ${\omega^{\prime},\omega^{\prime\prime}}$ such that}
    \[ f(\omega^{\prime})=\alpha,f(\omega^{\prime\prime})=\beta \]
    \item $\prv$ \textit{computes $w_2$ and $w_3$ following the above range proof and sends the commitment to $f$,} $\cm_f$
    \item $\vrf$ \textit{sends a random challenge $\gamma\in\mathbb{F}$}
    \item $\prv$ \textit{sends the commitment to $q_w:=w/(X^n-1)$ where}
    \[ w:=w_2+\gamma\cdot{w_3} \]
    \item $\vrf$ \textit{sends a random evaluation point $\zeta\in\mathbb{F}\setminus{H}$}
    \item $\prv$ \textit{sends the evaluations $f(\zeta),f(\zeta\omega),q_w(\zeta)$}
    \item $\prv$ \textit{sends the commitments to $q_1(X),q_2(X)$, where}
    \[ q_1(X):=\frac{f(X)-f(\zeta)}{X-\zeta}+\gamma\cdot\frac{q_w(X)-q_w(\zeta)}{X-\zeta} \]
    \[ q_2(X):=\frac{f(X)-f(\zeta\omega)}{X-\zeta\omega} \]
    \item $\vrf$ \textit{chooses random $r\in\mathbb{F}$}
    \item $\vrf$ \textit{outputs \textbf{acc} if and only if}
    \begin{enumerate}
    	\item $w_1(\zeta)+\gamma\cdot{w_2(\zeta)}=q_w(\zeta)\cdot(\zeta^n-1)$
    	\item $e(F+\zeta\cdot\cm_{q_1}+r\zeta\omega\cdot\cm_{q_2},[1]_2)=e(\cm_{q_1}+r\cdot\cm_{q_2},[x]_2)$\textit{, where}
    	\begin{align*}
    		F:=&\cm_f-[f(\zeta)]_1+\gamma\cdot(\cm_{q_w}-[q_w(\zeta)]_1)+r\cdot(\cm_f-[f(\zeta\omega)]_1)
    	\end{align*}
    \end{enumerate}
\end{enumerate}

\begin{theorem}
The above range proof with $\kzgzk$ is complete, sound, and HVZK.
\end{theorem}
\begin{proof}
Completeness follows the protocol. Soundness and zero knowledge can be verified by Plonk (Section 3.1 of~\cite{plonk}) and Theorem~\ref{thm:kzgzk}.
\end{proof}

\section{The Protocol ZKPoK in \bootstrap}
\label{alg:map}

Protocol~\ref{alg:zkpok} presents how to map the \textit{success} or \textit{failure} in \secp to \bls.

\FloatBarrier
% !TEX root = ../../main.tex

\begin{Protocol*}[t!]
\begin{mdframed}
$\prv$ and $\vrf$ are both given $\{\pb(A_{\mathsf{keys},i})\}$. $\prv$ has the access to $\{\mathsf{sk}_i\},\{A_{\mathsf{keys},i}\}$ and the hiding factor of $\pb(A_{\mathsf{keys},i})$, $r_i$.
\begin{enumerate}   
    \item Case 1: $A_{\mathsf{keys},i}=1$ ($\prv$ claims knowledge of $\mathsf{sk}_i$)
    \begin{enumerate}  
        \item $\prv$ selects $e_1\stackrel{\$}{\leftarrow} \{0,1\}^t; z_3,\beta\stackrel{\$}{\leftarrow}\mathbb{Z}_b;\alpha\stackrel{\$}{\leftarrow}\mathbb{Z}_s$
        \item $\prv$ publishes  $t_1=\gs^\alpha$
        \item $\prv$ publishes $t_2=\hb^\beta$
        \item $\prv$ publishes  $t_3=\gb^{-e_1}\hb^{z_3-r_ie_1}$
        \item $\vrf$ publishes $t$-bit challenge $e\stackrel{\$}{\leftarrow} \{0,1\}^t$ (or $\prv$ via Fiat-Shamir)
        \item $\prv$ computes $e_0=e\oplus{e_1}$ and publishes $e_0$ and $e_1$
        \item $\prv$ publishes $z_1=e_0\mathsf{sk}_i+\alpha$
        \item $\prv$ publishes $z_2=e_0r_i+\beta$
        \item $\prv$ publishes $z_3$
    \end{enumerate}
    
    \item Case 2: $A_{\mathsf{keys},i}=0$ ($\prv$ does not claim knowledge of $\mathsf{sk}_i$)
    \begin{enumerate}  
        \item $\prv$ selects $e_0\stackrel{\$}{\leftarrow} \{0,1\}^t;z_1\stackrel{\$}{\leftarrow}\mathbb{Z}_s;z_2,\alpha\stackrel{\$}{\leftarrow}\mathbb{Z}_b$
        \item $\prv$ publishes $t_1=\gs^{z_1}/\mathsf{pk}_i^{e_0}$
        \item $\prv$ publishes $t_2=\gb^{e_0}\hb^{z_2-r_ie_0}$
        \item $\prv$ publishes $t_3=\hb^\alpha$
        \item $\vrf$ publishes $t$-bit challenge $e\stackrel{\$}{\leftarrow} \{0,1\}^t$ (or $\prv$ via Fiat-Shamir)
        \item $\prv$ computes $e_1=e\oplus{e_0}$ and publishes $e_0$ and $e_1$    
        \item $\prv$ publishes $z_1$
        \item $\prv$ publishes $z_2$
        \item $\prv$ publishes $z_3=e_1r_i+\alpha$
    \end{enumerate}
    
    \item $\vrf$ outputs $\textbf{acc}$ if and only if
    \begin{enumerate}
        \item $e=e_0\oplus{e_1}$
        \item $\gs^{z_1}=\mathsf{pk}_i^{e_0}t_1$ 
        \item $\gb^{e_0}\hb^{z_2}=\pb(A_{\mathsf{keys},i})^{e_0}t_2$ 
        \item $\hb^{z_3}=\pb(A_{\mathsf{keys},i})^{e_1}t_3$
    \end{enumerate} 
\end{enumerate} 
\end{mdframed}
\caption{The ZKPoK proof demonstrates that $\prv$ can prove knowledge of a secret key with the correct committed selector.}\label{alg:zkpok}
\end{Protocol*}
    
\FloatBarrier

\section{Range Proof for Multiple Values}
\label{alg:rpmv}

Protocol~\ref{alg:multirange} presents how to prove multiple values satisfy the range proof based on binary decomposition.

\FloatBarrier
% !TEX root = ../../main.tex

% = = = 

\begin{Protocol*}[t!]
\begin{mdframed}
\footnotesize

% = = = 

\begin{enumerate}
    \item $\prv$ takes as input some values to be proved, $\{x_1,x_2,\dots,x_\mu\}$
    \item $\prv$ computes the binary decomposition (from most significant bit to least significant bit) of each balance, $\{z_j^{(x_i)}\}_{i\in[\mu],j\in[k]}$, such that $z_j^{(x_i)}\in\{0,1\}$ and $x_i=\sum_{j=k}^{0}2^j\cdot{z_j^{(x_i)}}$.
    \item $\prv$ puts the bits into accumulator form where $\chi_k^{(x_i)}=z_k^{(x_i)}$ and $\chi_i^{(x_i)}=2\chi_{i+1}^{(x_i)}+z_i^{(x_i)}$.  (Remark: visualized as a matrix, each row is a balance where the $k$-th column is the least significant bit and, moving right-to-left, each bit is folded in until it accumulates to $x_j$ in the first column.)
    \[\begin{bmatrix}
        \chi_1^{(x_1)} & \chi_2^{(x_1)} & \chi_3^{(x_1)} & \dots & \chi_k^{(x_1)} \\[3pt]
        \chi_1^{(x_2)} & \chi_2^{(x_2)} & \chi_3^{(x_2)} & \dots & \chi_k^{(x_2)} \\[3pt]
        \chi_1^{(x_3)} & \chi_2^{(x_3)} & \chi_3^{(x_3)} & \dots & \chi_k^{(x_3)} \\[3pt]
        \vdots & \vdots & \vdots & \ddots & \vdots \\[3pt]
        \chi_1^{(x_\mu)} & \chi_2^{(x_\mu)} & \chi_3^{(x_\mu)} & \dots & \chi_k^{(x_\mu)}
    \end{bmatrix}\]
    \item Due to the concrete parameters of \bls, $\prv$ will work column-by-column (proof size and verifier time will be linear in $k$ which is the bit-precision of each account). Let column $j$ be vector $\overline{p}_j=\{\chi_j^{(x_1)},\chi_j^{(x_2)},\dots,\chi_j^{(x_\mu)}\}$. The following constraints apply (for $i\in[\mu],j\in[k]$):       $\overline{p}_1[i]=x_i$; $\overline{p}_j[i] - 2\cdot\overline{p}_{j+1}[i]\in\{0,1\}$; and $\overline{p}_k[i]\in\{0,1\}$. $\overline{p}_1$ contains $\{x_1,x_2,\dots,x_\mu\}$.
	\item $\prv$ interpolates polynomials for $\overline{p}_j \rightarrow p_j(X)$ and publishes commitments to each.
    \item $\prv$ shows the following polynomials are vanishing for all $x\in{H}$ where $H=\{\omega^0,\omega^1,\dots,\omega^{k-1}\}$
    \begin{align*}
        v_1:=&[p_1(X)-2p_2(X)]\cdot[1-(p_1(X)-2p_2(X))] \\
        v_2:=&[p_2(X)-2p_3(X)]\cdot[1-(p_2(X)-2p_3(X))] \\
        \vdots \\
        v_{k-1}:=&[p_{k-1}(X)-2p_k(X)]\cdot[1-(p_{k-1}(X)-2p_k(X))] \\
        v_k:=&p_k(X)\cdot[1-p_k(X)]
    \end{align*}
    $\{v_1,v_2,\dots,v_k\}$ prove each liability is greater than or equal to 0 (range proof). To complete the proof, $\prv$ and $\vrf$ run $\kzgzk$ to open $(p_1,p_2,\dots,p_k)$ at a random evaluation point.
    \item $\vrf$ outputs \textbf{acc} if and only if
    \begin{enumerate}
        \item each evaluation is valid
        \item $\{v_1,v_2,\dots,v_k\}$ are vanishing on $H$
    \end{enumerate}
\end{enumerate}

% = = = 

\normalsize	
\end{mdframed}
\caption{The range proof for multiple values demonstrates that each value is either zero or a positive number less than a specified value. \label{alg:multirange}}
\end{Protocol*}

\FloatBarrier
