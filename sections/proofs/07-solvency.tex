% !TEX root = ../../main.tex

% = = = 

\begin{Protocol*}[t!]
\begin{mdframed}
\footnotesize

% = = = 

\begin{enumerate}
    \item $\prv$ computes equality $\mathsf{eq}$ as the total assets minus the total liabilities. 
    \item $\prv$ publishes commitment to polynomial $f_\mathsf{eq}(X)$ where $f_\mathsf{eq}(\omega^0)=\mathsf{eq}$.
    \item $\prv$ generates a range proof for $\mathsf{eq}$ in $f_\mathsf{eq}(X)$ to demonstrate it is a non-negative integer.
    \item $\prv$ opens $f_\mathsf{eq}(\omega^0)$ through $\kzgcm$ and publishes $\pb(f_\mathsf{assets}(\omega^0))$, $\pb(f_\mathsf{liab}(\omega^0))$ from \poa and \pol.
    \item $\vrf$ outputs \textbf{acc} if and only if
    \begin{enumerate}
        \item $\pb(f_\mathsf{assets}(\omega^0))=\pb(f_\mathsf{liab}(\omega^0))\cdot\pb(f_\mathsf{eq}(\omega^0))$.
        \item The range proof for $\mathsf{eq}$ is valid.
    \end{enumerate}
\end{enumerate}

% = = = 

\normalsize	
\end{mdframed}
\caption{The \pos proof demonstrates that the total assets exceed the total liabilities by a non-negative integer (called the equity). \label{alg:pos}}
\end{Protocol*}
