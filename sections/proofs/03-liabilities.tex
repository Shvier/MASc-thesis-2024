% !TEX root = ../../main.tex

% = = = 

\begin{Protocol*}[t!]
\begin{framed}
\footnotesize

% = = = 

$\prv$'s private input: user balances, $\{\bal_1,\bal_2,\dots,\bal_\mu\}$
\begin{enumerate}
        \item $\prv$ computes the binary decomposition (from most significant bit to least significant bit) of each balance, $\{z_j^{(\bal_i)}\}_{i\in[\mu],j\in[k]}$, such that $z_j^{(\bal_i)}\in\{0,1\}$ and $\bal_i=\sum_{j=k}^{0}2^j\cdot{z_j^{(\bal_i)}}$
%        \item Flat map the bits of each balance into a matrix
%        \[\begin{bmatrix}
%            z_1^{(\bal_1)} & z_2^{(\bal_1)} & z_3^{(\bal_1)} & \dots & z_m^{(\bal_1)} \\[3pt]
%            z_1^{(\bal_2)} & z_2^{(\bal_2)} & z_3^{(\bal_2)} & \dots & z_m^{(\bal_2)} \\[3pt]
%            z_1^{(\bal_3)} & z_2^{(\bal_3)} & z_3^{(\bal_3)} & \dots & z_m^{(\bal_3)} \\[3pt]
%            \vdots & \vdots & \vdots & \vdots & \vdots \\[3pt]
%            z_1^{(\bal_k)} & z_2^{(\bal_k)} & z_3^{(\bal_k)} & \dots & z_m^{(\bal_k)}
%        \end{bmatrix}\]
        \item $\prv$ puts the bits into accumulator form where $\chi_k^{(\bal_i)}=z_k^{(\bal_i)}$ and $\chi_i^{(\bal_i)}=2\chi_{i+1}^{(\bal_i)}+z_i^{(\bal_i)}$.  (Remark: visualized as a matrix, each row is a balance where the $k$-th column is the least significant bit and, moving right-to-left, each bit is folded in until it accumulates to $\bal_j$ in the first column.)
        \[\begin{bmatrix}
            \chi_1^{(\bal_1)} & \chi_2^{(\bal_1)} & \chi_3^{(\bal_1)} & \dots & \chi_k^{(\bal_1)} \\[3pt]
            \chi_1^{(\bal_2)} & \chi_2^{(\bal_2)} & \chi_3^{(\bal_2)} & \dots & \chi_k^{(\bal_2)} \\[3pt]
            \chi_1^{(\bal_3)} & \chi_2^{(\bal_3)} & \chi_3^{(\bal_3)} & \dots & \chi_k^{(\bal_3)} \\[3pt]
            \vdots & \vdots & \vdots & \vdots & \vdots \\[3pt]
            \chi_1^{(\bal_\mu)} & \chi_2^{(\bal_\mu)} & \chi_3^{(\bal_\mu)} & \dots & \chi_k^{(\bal_\mu)}
        \end{bmatrix}\]
        
        \item Due to the concrete parameters of \bls, $\prv$ will work column-by-column (proof size and verifier time will be linear in $k$ which is the bit-precision of each account). Let column $j$ be vector $\overline{p}_j=\{\chi_j^{(\bal_1)},\chi_j^{(\bal_2)},\dots,\chi_j^{(\bal_\mu)}\}$. The following constraints apply (for $i\in[\mu],j\in[k]$):       $\overline{p}_1[i]=\bal_i$; $\overline{p}_j[i] - 2\cdot\overline{p}_{j+1}[i]\in\{0,1\}$; and $\overline{p}_k[i]\in\{0,1\}$. $\overline{p}_1$ contains $\{\bal_1,\bal_2,\dots,\bal_\mu\}$.
        
        
        \item  $\prv$ builds an additive accumulator $\nu$ for $\overline{p}_1$ where $\nu_k=\bal_k=\overline{p}_1[k]$ and $\nu_i=\nu_{i+1}+\bal_i,i\in[1,\mu]$. Remark: $\nu_1$ will contain the total liability balances. 
        
	 \item $\prv$ interpolates polynomials for $\overline{p}_j \rightarrow p_j(X)$ and publishes commitments to each using KZG in extended zero-knowledge form (see Section~\ref{sec:kgzzkp}).
 
	
        \item $\prv$ shows the following polynomials are vanishing for all $x\in{H}$ where $H=\{\omega^0,\omega^1,\dots,\omega^{k-1}\}$
        \begin{align*}
            w_1:=&[p_0(X)-p_0(X\omega)-p_1(X)]\cdot(X-\omega^{\mu-1}) \\
            w_2:=&[p_0(X)-p_1(X)]\cdot\frac{X^\mu-1}{X-\omega^{\mu-1}} \\
            w_3:=&p_m(X)\cdot[1-p_m(X)] \\
            v_1:=&[p_1(X)-2p_2(X)]\cdot[1-(p_1(X)-2p_2(X))] \\
            v_2:=&[p_2(X)-2p_3(X)]\cdot[1-(p_2(X)-2p_3(X))] \\
            \vdots \\
            v_{k-1}:=&[p_{k-1}(X)-2p_k(X)]\cdot[1-(p_{k-1}(X)-2p_k(X))]
        \end{align*}
        $w_1$ and $w_2$ prove the accumulative vector is correct, $w_3$ and $\{v_i\}$ prove each liability is greater than or equal to 0 (range proof). To complete the proof, $\prv$ will run $\open(p_0,p_1,p_2,\dots,p_k)$ at a random evaluation point $\zeta$ and $\open(p_0)$ at $\zeta\omega$
        % This step involves the verifier sampling a random point as the challenge and the prover responding with the evaluations and the alleged commitments of each polynomial $p_i$ at that point.
        % \begin{enumerate}
        %     \item $\prv$ generates two random numbers $\omega^{\prime},\omega^{\prime\prime}$ in $\mathbb{F}\setminus{H}$ and another two random numbers $\alpha,\beta$ in $\mathbb{F}$ for each polynomial $p_i$
        %     \item $\prv$ interpolates $\{p_i\}$ at two more points ${\omega_i^{\prime},\omega_i^{\prime\prime}}$ where
        %     \[ p_i(\omega_i^{\prime})=\alpha_i \]
        %     \[ p_i(\omega_i^{\prime\prime})=\beta_i \]
        %     Note this should be done at the beginning of the protocol
        %     \item $\prv$ sends all commitments of ${p_i}$
        %     \[ \{\cm_{p_0},\cm_{p_1},\dots,\cm_{p_m}\} \]
        %     \item $\vrf$ sends a random evaluation point $\tau\in\mathbb{F}\setminus{H}$
        %     \item $\prv$ sends the evaluations $\{y_i\}$ of $\{p_i(\tau)\}$ and $p_0(\tau\omega)$
        %     \item $\vrf$ sends a random challenge $\gamma\in\mathbb{F}$
        %     \item $\prv$ computes
        %     \[ w:=w_1+\gamma w_2+\gamma^2 w_3+\sum_{i=1}^{m-1}\gamma^{i+2}v_i \]
        %     and sends the commitment to the quotient polynomial $[q_w]_1$ where
        %     \[ q_w:=w/(X^k-1) \]
        %     \item Let $\hat{w}:=w-q_w\cdot(X^k-1)$. Note $\hat{w}$ is a zero polynomial for all $x\in\mathbb{F}$, which means $X-\tau$ divides $\hat{w}$. Then $\prv$ sends $[q_{\hat{w}}]_1$ where
        %     \[ q_{\hat{w}}:=\hat{w}/(X-\tau) \]
        %     \item $\vrf$ accepts the proof if and only if $e(F,[1]_2)=e([q_{\hat{w}}]_1,[x-\tau]_2)$, where
        %     \begin{align*}
        %         F:=&(\cm_{p_0}-[p_0(\tau\omega)+p_1(\tau)]_1)\cdot(\tau-\omega^{k-1}) \\
        %          & + \gamma\cdot(\cm_{p_0}-[p_1(\tau)]_1)\cdot{\frac{\tau^k-1}{\tau-\omega^{k-1}}} \\
        %          & + \gamma^2\cdot{\cm_{p_m}-[p_m(\tau)]_1} \\
        %          & + \sum_{i\in[1,m-1]}\gamma^{i+2}\cdot[(\cm_{p_i}-[2p_{i+1}(\tau)]_1)\cdot(\cm_{p_i}-[2p_{i+1}(\tau)+1]_1)] \\
        %          & - [q_w]_1\cdot(\tau^n-1)
        %     \end{align*}
        % \end{enumerate}
    %\item $\prv$ reveals the commitments to $\{p_i\}$ from each group and the committed liablity $\pb(\liab)$
\end{enumerate}

$\vrf$ protocol:
\begin{enumerate}
    \item Verify each evaluation is valid
    \item Verify $\{w_1,w_2,w_3\}$ and $\{v_i\}$ are vanishing in $H$
\end{enumerate}




% = = = 

\normalsize	
\end{framed}
\caption{The \pol proof demonstrates that each liabilities is either zero or a positive number, and that the balances are totalled correctly in $\pi_1(\omega^0)$. \label{alg:pol}}
\end{Protocol*}
