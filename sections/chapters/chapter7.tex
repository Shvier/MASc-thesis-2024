\chapter{Concluding Remarks}

We presented \Sys as a Poly-IOP solvency argument. The efficiency and succinctness of \Sys might be further improved using advances in other poly-iop systems: lookup arguments and multivariate polynomials (and corresponding commitment schemes). Techniques from recursive SNARKs might enable solvency proofs that are repeated (say every day) to reduce prover time by ignoring ETH addresses and user balances that did not change across the day, while still allowing a verifier to have full confidence in the history of the exchange if they only check the most recent proof. If blockchains like Ethereum add low-gas cost support for \bls, a topic of discussion (EIP-2537\footnote{\url{https://eips.ethereum.org/EIPS/eip-2537}}), verifying proofs of solvency could move on-chain. If an exchange fails to provide a smart contract with a proof of solvency in a timely fashion, the smart contract could be called to trigger penalties or other actions. 

Standardization work could also be useful for proofs of solvency. Of the exchanges that opt into providing proofs of solvency, the exact protocol can vary from other exchanges, and it is not always deployed correctly (at least initially)~\cite{broken}. Having the community coalesce behind a proof template, working out every detail, with audited and formally verified reference code could be helpful to exchanges. \Sys is a good start as it is a complete end-to-end proof system, and is competitive in prover time, verifier time and proof size with other sub-components introduced in the literature.
