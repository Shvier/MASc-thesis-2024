\chapter{Cryptographic Building Blocks}

\section{Range Proof}
\label{sec:range}

A range proof enables $\prv$ to convince $\vrf$ a value $x$ is in a specified range, e.g., $[0,2^k)$, without revealing $x$. Zero-knowledge range proofs (ZKRPs) have three typical approaches: square decomposition, $n$-ary decomposition, and hash chains~\cite{zkrp}. We use the polynomial-based range proof from Boneh \etal~\cite{rangeproof}.
\begin{enumerate}
    \item \textit{Given input $x$, $\prv$ decomposes $x$ to a vector of binary digits $\overline{z}=\tuple{z_1,z_2,\dots,z_k}$, so that $x=\sum_{i=0}^{k-1}2^i\cdot{z_i}$} 
    \item \textit{$\prv$ constructs a vector $\overline{x}=\tuple{x_1,x_2,\dots,x_k}$ such that}
    \begin{align*}
        x_1&=x \\
        x_k&=z_k \\
        x_i&=2x_{i+1}+z_i,i\in[1,k-1]
    \end{align*}
    \item \textit{$\prv$ interpolates a polynomial $f$ from $\overline{x}$ over a finite field $H$ of order $n$ with elements $\omega^0,\omega^1,\omega^2,\ldots,\omega^{n-1}$} 
    \item \textit{$\prv$ proves the following polynomials are vanishing in $H$}
    \begin{align*}
        w_1&:=[f(X)-x]\cdot\frac{X^n-1}{X-\omega^0} \\
        w_2&:=f(X)\cdot[f(X)-1]\cdot\frac{X^n-1}{X-\omega^{n-1}} \\
        w_3&:=[f(X)-2\cdot{f(X\omega)}]\cdot[f(X)-2\cdot{f(X\omega)}-1]\cdot(X-\omega^{n-1})
    \end{align*}
    \begin{enumerate}
        \item $\prv$ \textit{sends the commitment to $f(X)$}
        \item $\vrf$ \textit{sends a random challenge $\gamma$}
        \item $\prv$ \textit{sends the commitment to $q(X)=w/(X^n-1)$, such that}
        \[ w=w_1+\gamma\cdot{w_2}+\gamma^2\cdot{w_3} \]
        \item $\vrf$ \textit{sends a random evaluation point $\zeta\in\mathbb{F}$}
        \item $\prv$ \textit{replies with $f(\zeta),f(\zeta\omega),q(\zeta)$}
        \item $\vrf$ \textit{checks}
        \begin{enumerate}
            \item $w(\zeta)=q(\zeta)\cdot(\zeta^n-1)$
            \item $f(\zeta),f(\zeta\omega),q(\zeta)$ \textit{are the correct evaluations through the verifying process of KZG}
        \end{enumerate}
    \end{enumerate}
\end{enumerate}

\begin{lemma}
\label{lemma:range}
The range proof from Boneh \etal~\cite{rangeproof} is complete and has knowledge soundness in the algebraic group model.
\end{lemma}

\begin{proof}
Completeness is clear by following the protocol.

For knowledge soundness, to make the equation $w(\zeta)=q(\zeta)\cdot(\zeta^n-1)$ hold, $q(X)$ must exist (it is a rational function). That means $w(X)$ is vanishing on $H$, i.e., $w_1,w_2$, and $w_3$ are vanishing over $H$. Thus, if $f(X)$ does not satisfy any of the equations $w_1,w_2$, and $w_3$, $\vrf$ will detect the proof is invalid. By the binding property of KZG commitment, we know that the evaluations $f(\zeta),f(\zeta\omega)$, and $q(\zeta)$ are correct with overwhelmingly high probability if the KZG verifying is passed.
\end{proof}
We note that the above range proof is not zero-knowledge because (i) revealing $f(X)$ and $f(X\omega)$ leaks the evaluations of $f$ and (ii) verifying $w_1$ requires $\vrf$ to know $x$. The solutions for these two points may seem similar but not quite the same, because the first point requires $\prv$ not to open the evaluations of $f(X)$ over $H$, and the second point means $\prv$ is supposed to reveal the evaluation of $f(\omega^0)$ to prove $x$ is the claimed one. Boneh \etal~\cite{rangeproof} introduced a zero-knowledge extension to solve the first issue. We will describe the extension and manipulate a variant of the opening scheme to solve the second one.

\section{Batched Opening with Zero-Knowledge Extension}
\label{sec:kgzzkp}
To efficiently prove several polynomials are vanishing at several points, there are some batched KZG opening schemes such as the schemes in \cite{plonk,bdfg,fflonk}. Here, we use the batched opening scheme from~\cite{plonk} with the zero-knowledge extension from~\cite{rangeproof} to explain how to prove the range-proof polynomials are vanishing efficiently and in zero-knowledge. \\
\textit{Assume $\prv$ is given $x$ and $\prv$ computes $f(X)$ using the above range proof. $\prv$ wants to prove $f(X)$ satisfies the second and the third condition, i.e., $w_2$ and $w_3$ are vanishing over $H$:}
\begin{enumerate}
    \item $\prv$ \textit{generates two random numbers $\omega^{\prime},\omega^{\prime\prime}\in\mathbb{F}\setminus{H}$ and another two random numbers $\alpha,\beta\in\mathbb{F}$}
    \item $\prv$ \textit{interpolates $f$ at two more points ${\omega^{\prime},\omega^{\prime\prime}}$ such that}
    \[ f(\omega^{\prime})=\alpha \]
    \[ f(\omega^{\prime\prime})=\beta \]
    \textit{The new polynomial is called $f^\prime$.}
    \item $\prv$ \textit{computes $w_2$ and $w_3$ following the above range proof and sends the commitment to $f^\prime$,} $\cm_{f^\prime}$
    \item $\vrf$ \textit{sends a random challenge $\gamma\in\mathbb{F}$}
    \item $\prv$ \textit{sends the commitment to $q_w:=w/(X^n-1)$ where}
    \[ w:=w_2+\gamma\cdot{w_3} \]
    \item $\vrf$ \textit{sends a random evaluation point $\zeta\in\mathbb{F}\setminus{H}$}
    \item $\prv$ \textit{sends the evaluations $f^\prime(\zeta),f^\prime(\zeta\omega),q_w(\zeta)$}
    \item $\prv$ \textit{sends the commitments to $q_1(X),q_2(X)$, where}
    \[ q_1(X):=\frac{f^\prime(X)-f^\prime(\zeta)}{X-\zeta}+\gamma\cdot\frac{q_w(X)-q_w(\zeta)}{X-\zeta} \]
    \[ q_2(X):=\frac{f^\prime(X)-f^\prime(\zeta\omega)}{X-\zeta\omega} \]
    \item $\vrf$ \textit{chooses random $r\in\mathbb{F}$}
    \item $\vrf$ \textit{accepts the proof if and only if}
    \begin{enumerate}
    	\item $w_1(\zeta)+\gamma\cdot{w_2(\zeta)}=q_w(\zeta)\cdot(\zeta^n-1)$
    	\item $e(F+\zeta\cdot\cm_{q_1}+r\zeta\omega\cdot\cm_{q_2},[1]_2)=e(\cm_{q_1}+r\cdot\cm_{q_2},[x]_2)$\textit{, where}
    	\begin{align*}
    		F:=&\cm_{f^\prime}-[f^\prime(\zeta)]_1+\gamma\cdot(\cm_{q_w}-[q_w(\zeta)]_1)+r\cdot(\cm_{f^\prime}-[f^\prime(\zeta\omega)]_1)
    	\end{align*}
    \end{enumerate}
\end{enumerate}

\begin{theorem}
The batched KZG opening scheme with the zero-knowledge extension is complete, sound, and HVZK.
\end{theorem}
\begin{proof}
Completeness is clear by following the protocol. Soundness can be verified by Lemma~\ref{lemma:range} and~\cite{plonk}. \\
To verify zero knowledge, we construct a simulator $\mathcal{S}$. Let $\mathcal{S}$ randomly generate $\{\omega^{*^\prime},\omega^{*^{\prime\prime}},\alpha^*,\beta^*\}$ like $\prv$, and interpolate $f^*$ such that
\[ f^*(\omega^{*^\prime})=\alpha^* \]
\[ f^*(\omega^{*^{\prime\prime}})=\beta^* \]
\[ f^*(x)=0,x\in{H} \]
We can observe when $\vrf$ interacts with $\mathcal{S}$ to execute the protocol, $\vrf$ always accepts the proof from $\mathcal{S}$. Given $\{\omega^{*^\prime},\omega^{*^{\prime\prime}},\alpha^*,\beta^*\}$ are chosen uniformly at random each time, that is excatly the same as $\prv$ interpolates $f^\prime$. It should be clear $\vrf$ cannot distinguish between the transcript from $\mathcal{S}$ and the transcript from $\prv$.
\end{proof}
We will use $\openzk$ to denote this opening technique.

\section{Open KZG with Committed Value}
\label{sec:kzgOpenComm}
By Definition~\ref{def:pcs}, to prove $b$ is the evaluation of $f(a)$, $\prv$ reveals the pair $(a,b)$ to let $\vrf$ validate the proof through pairings. To solve the second issue in the above range proof, we describe the new opening scheme.
\begin{definition}[$\openc$]
$\openc$ is a KZG opening scheme that $\prv$ is given input $f$ and $b$. $\prv$ and $\vrf$ are both given
\begin{itemize}
    \item \textbf{srs} = $\tuple{g_1,g_1^\tau,\dots,g_1^{\tau^d},h_1,h_1^\tau,\dots,h_1^{\tau^d},g_2,g_2^\tau}$
    \item $\cm$ - the commitment to $f$
    \item $a$ - an evaluation point of $f$
    \item $\textbf{C}(b)$ - the committed evaluation of $f(a)$, $g_1^{f(a)}h_1^{\hat{f}(a)}$
\end{itemize}
\textit{They run the protocol as follows:}
\begin{enumerate}
    \item $\prv$ computes the witness $w$ for $(a,b,\hat{b})$ such that
    \[ w=g_1^{\psi(\tau)}h_1^{\hat\psi(\tau)} \]
    where $\psi(x)=\frac{f(X)-f(a)}{X-a}$, and $\hat\psi(x)=\frac{\hat{f}(X)-\hat{f}(a)}{X-a}$
    \item $\vrf$ outputs \textbf{acc} if and only if
    \[ e(\cm/\textbf{C}(b),[1]_2)=e(w,[\tau-a]_2) \]
\end{enumerate}
\end{definition}
\begin{theorem}
\label{thm:kzgOpen}
$\openc$ is complete, sound, and special HVZK.
\end{theorem}
\begin{proof}
Completeness follows the original KZG commitment scheme.

For soundness, $\openc$ does not violate the soundness of the original KZG commitment scheme. Note $\textbf{C}(b)$ is a Pedersen commitment. Recall the computational binding property of Pedersen commitment, that means it is infeasible for $\prv$ to compute a $b^*$ such that $f(a)\ne{b^*},\textbf{C}(b)=\textbf{C}(b^*)$ based on discrete logarithm assumption.

To prove special HVZK, let the simulator $\mathcal{S}$ take as inputs $\textbf{srs}$, $\cm_f$, $a$, and $\textbf{C}(b)$. Since $\mathcal{S}$ knows the random evaluation point $a$ in advance, $\mathcal{S}$ can compute a witness $w^*=\cm_f/\textbf{C}(b)/[\tau-a]_1$ that makes the pairing equation hold, i.e., $\mathcal{S}$ always outputs an accepting conversation $(\cm_f,a,w^*)$ for $\textbf{C}(b)$.
\end{proof}
Back to the above range proof, now $\prv$ is able to prove $w_1$ is correct with $\openc$ in zero-knowledge while using $\openzk$ to prove $w_2$ and $w_3$ are correct. Together, we can prove a value $x$ is in the specified range without revealing $x$. For simplicity, we use $\open(f,a,b)$ to denote opening the evaluation $b$ of $f(a)$, where $\open$ means $\openzk$ if $b$ is a plain value and $\openc$ if $b$ is a committed value.
