\chapter{Proof of Solvency}

\section{Proof of Assets (PoA)}

We will begin on the asset-side of the proof of solvency. To ease the writing, we will take the example of an exchange proving solvency for ETH on Ethereum. The PoA is broken into two steps: \bootstrap and \poa. \bootstrap takes the following public (exchange-chosen) input: a set of Ethereum public keys which consists of its own public keys hidden amongst a larger set of keys belonging to others (\ie an anonymity set which can scale to the full set of public keys on Ethereum); private input: set of private signing keys (in \secp); and outputs `the keys that belong to it.' The exact format of the output can vary; in our case, the input keys are indexed and the output is a binary `selector' vector (in \bls) that records a 0 if the exchange is not claiming to know the secret key and a 1 if the exchange can prove knowledge of the secret key. The main alternative format would be a flat list of the known public keys which can serve as a set for set-membership proofs, which can be performed succinctly with lookup arguments (see IZPR for this approach~\cite{izpr} where it is called `bootstrapping'). Presently, \bootstrap with such an output format is only known to be realizable by general zk-SNARK circuits, whereas we provide direct proof technique for a selector vector output format.

Once the \bootstrap output is provided, \poa takes it as input, along with the current balance of ETH in each address in the anonymity set (publicly known on Ethereum's blockchain). The output is a commitment to the total assets across keys and an argument the total is computed correctly. 

% = = = = = = = = = = = = = = = = = = = = = = = = = = = = = = = = = = =

\subsection{The \bootstrap Proof}
% !TEX root = ../../main.tex

\begin{Protocol*}[t!]
    \begin{framed}
    \footnotesize
    \begin{enumerate}
    \item $\prv$ publishes an anonymity set of public keys: $\tuple{\mathsf{pk}_1,\mathsf{pk}_2,\mathsf{pk}_3,\dots,\mathsf{pk}_n}$.
    \item $\prv$ constructs a selector vector: $\overline{s}=\tuple{s_1,s_2,s_3,\ldots,s_n}$ where $s_i=0$ unless $\prv$ can prove knowledge of $\mathsf{sk}_i$ given $\mathsf{pk}_i=\gs^{\mathsf{sk}_i}$, then  $s_i=1$.
    \item $\prv$ interpolates a polynomial $\phi(X)$ from $\overline{s}$, and runs the following steps
    \begin{enumerate}
        \item $\prv$ computes the KZG polynomial commitment to $\phi$. Specifically, $\prv$ constructs a random polynomial $\hat\phi(X)$ and publishes the polynomial commitment $\cm_\phi$ such that $\cm_\phi=\gb^{\phi(\tau)}\hb^{\hat\phi(\tau)}$.
        \item For each $i$, $\prv$ runs $\open(\phi, \omega^i, p_i)$, where $p_i=\gb^{\phi(\omega^i)}\hb^{\hat\phi(\omega^i)}$.
        \item $\prv$ publishes the witness for each evaluation, $w_i$.
        \item $\vrf$ verifies $\{p_i\}$ are the evaluations of $\phi$ by checking $e(\cm_\phi/p_i, [1]_2)=e(w_i, [\tau-\omega^i]_2)$.
    \end{enumerate}
    \item For each $p_i$, $\prv$ and $\vrf$ run $\mathrm{ZKPoK} \{ (sk_i,s_i) :  [\mathsf{pk}_i=g^{\mathsf{sk}_i} \land p_i=\Comm{1,r_i}  ] \lor p_i=\Comm{0,r_i} \}$ as follows:
    
    \begin{enumerate}   
    \begin{framed}
        \item Case 1: $s_i=1$ ($\prv$ claims knowledge of $sk_i$)
        \begin{itemize}  
        \item $\prv$ selects $e_1\stackrel{\$}{\leftarrow} \{0,1\}^t; z_3,\beta\stackrel{\$}{\leftarrow}\mathbb{Z}_b;\alpha\stackrel{\$}{\leftarrow}\mathbb{Z}_s$
        \item $\prv$ publishes  $t_1=\gs^\alpha$
        \item $\prv$ publishes $t_2=\hb^\beta$
        \item $\prv$ publishes  $t_3=\gb^{-e_1}\hb^{z_3-r_ie_1}$
        \item $\vrf$ publishes $t$-bit challenge $e\stackrel{\$}{\leftarrow} \{0,1\}^t$ (or $\prv$ via Fiat-Shamir)
        \item $\prv$ computes $e_0=e\oplus{e_1}$ and publishes $e_0$ and $e_1$
        \item $\prv$ publishes $z_1=e_0\mathsf{sk}_i+\alpha$
        \item $\prv$ publishes $z_2=e_0r_i+\beta$
        \item $\prv$ publishes $z_3$
        \end{itemize}
    
        \item Case 2: $s_i=0$ ($\prv$ does not claim knowledge of $sk_i$)
        \begin{itemize}  
        \item $\prv$ selects $e_0\stackrel{\$}{\leftarrow} \{0,1\}^t;z_1\stackrel{\$}{\leftarrow}\mathbb{Z}_s;z_2,\alpha\stackrel{\$}{\leftarrow}\mathbb{Z}_b$
        \item $\prv$ publishes $t_1=\gs^{z_1}/\mathsf{pk}_i^{e_0}$
        \item $\prv$ publishes $t_2=\gb^{e_0}\hb^{z_2-r_ie_0}$
        \item $\prv$ publishes $t_3=\hb^\alpha$
        \item $\vrf$ publishes $t$-bit challenge $e\stackrel{\$}{\leftarrow} \{0,1\}^t$ (or $\prv$ via Fiat-Shamir)
        \item $\prv$ computes $e_1=e\oplus{e_0}$ and publishes $e_0$ and $e_1$    
        \item $\prv$ publishes $z_1$
        \item $\prv$ publishes $z_2$
        \item $\prv$ publishes $z_3=e_1r_i+\alpha$
        \end{itemize}
    
        \item $\vrf$ checks
        \begin{itemize}
        \item $e=e_0\oplus{e_1}$
        \item $\gs^{z_1}=\mathsf{pk}_i^{e_0}t_1$ 
        \item $\gb^{e_0}\hb^{z_2}=p_i^{e_0}t_2$ 
        \item $\hb^{z_3}=p_i^{e_1}t_3$
        \end{itemize}  
    \end{framed}
    \end{enumerate}
    
    \end{enumerate}
    \normalsize	
    \end{framed}
    \caption{The \bootstrap proof demonstrates that $\phi(X)$ encodes a binary selector vector of the public keys for which the exchange can prove knowledge of the corresponding secret key. \label{alg:boot}}
    \end{Protocol*}
    

Before presenting our \bootstrap proof, we quickly discuss a few approaches that helped us develop it. A highly relevant $\Sigma$-protocol from the literature, COPZ, proves that two commitments in two different groups (\eg \secp and \bls) commit to the same value~\cite{chase22}. The paper cites proof of assets as a use-case but does not work out a protocol. COPZ allows an exchange to `map' private keys from \secp to \bls. Two places this mapping could occur would be at the very start or the very end of the assets proof. At the end, it might look like this: an existing proof of assets protocol (\eg Provisions~\cite{provisions}) could be run to create a commitment to the total assets in \secp, then COPZ can be used to prove the same commitment in \bls, and finally this can be `glued' to a succinct proof of liabilities in \bls. However this does not leverage the fact that \bls might help make the assets proof succinct.

Alternatively, the exchange can map all their keys at the start. There is a roadblock: the exchange can only map keys they know the secret key for and the exchange cannot reveal which keys they know and which they do not. Assume there is a protocol that would allow the exchange to output a sparse vector of \bls private key values (sorted by index of known ETH public keys) containing the key value when they know it, and recording a 0 if they do not. We designed such a protocol only to realize that the key values in \bls are actually never used, we only use the fact that knowledge of them is proven (which is covered by the ability to produce the value in \bls) and the fact that unclaimed keys are zeroed-out. 

This leads to our key observation: we do not need to map \textit{values} from \secp to \bls, we just have to map the \textit{success or failure} of a ZKP in \secp to \bls. This can be accomplished by composing (conjunction and disjunction of) $\Sigma$-protocols. The prover (exchange) will choose set of Ethereum public keys as its anonymity set of size $\kappa$ (containing its actual keys) $\{\mathsf{pk}_1,\mathsf{pk}_2,\dots,\mathsf{pk}_\kappa\}$ and publish them in an ordered (indexed) way. It will also create a binary `selector' array $A_\mathsf{keys}$ with a 1 in the same index of every key it is claiming to know and a 0 in the index of the keys it does not know (or does not want to claim for whatever reason). This vector is interpolated into the evaluation points of a polynomial $P_\mathsf{keys}(X)$  and committed to $C_\mathsf{keys}=\Comm{P_\mathsf{keys}(X)}$ using the KZG polynomial commitment scheme~\cite{kzg}. For each index $i$, the prover shows the evaluation of $P_\mathsf{keys}(X_i)$ but instead of providing the evaluation value ($A_{\mathsf{keys},i}$) in plaintext, it provides a Pedersen commitment to it $\Comm{A_{\mathsf{keys},i}}$ (a mild modification of the KZG showing protocol detailed in Section~\ref{sec:kzgOpenComm}). It then shows the value is correct with the following $\Sigma$-protocol:

\[ \mathrm{ZKPoK} \{ (sk_i,A_{\mathsf{keys},i}) : \Comm{A_{\mathsf{keys},i}}=\Comm{0} \lor [\Comm{A_{\mathsf{keys},i}}=\Comm{1} \land \mathsf{pk}_i=g^{\mathsf{sk}_i} ] \} \]

In plain English, the prover either: (1) puts a 0 in the selector vector; or (2a) puts a 1 in the selection vector \textit{and} (2b) knows the private key of the given public key. (1) and (2b) are a PoK of a representation for Pedersen commitments in \bls while (2b) is a Schnorr PoK of a discrete logarithm in \secp---both well studied $\Sigma$-protocols~\cite{damgard10,sigma}. The fact that (1) and (2b) are in \bls while (2c) is in \secp is not problematic because the disjunction ($\lor$) and conjunction ($\land$) operations for composing $\Sigma$-protocols are based only on how challenge values are constructed and both groups (\secp and \bls) can encode a large $t$-bit challenge (\eg $t=254$) into their exponent groups. 

As this protocol is repeated for each key, it is not succinct and will be $O(\kappa)$ in proof size and verifier time. However once the selector array is proven correct, the exchange can re-use it every time it does a proof of solvency until it updates its keys. The full details are provided in Protocol~\ref{alg:boot}.

% = = = = = = = = = = = = = = = = = = = = = = = = = = = = = = = = = = =

\subsection{The \poa Argument}
% !TEX root = ../../main.tex

% = = = 

\begin{Protocol*}[t!]
\begin{framed}
\footnotesize

% = = = 

\begin{enumerate}
	\item $\vrf$ processes the balance of the anonymous set
	\begin{enumerate}
		\item $\vrf$ queries the balance of each public key in the set $\prv$ chose, denoted by $\overline{b}=\tuple{\bal_1,\bal_2,\dots,\bal_n}$
		\item $\vrf$ interpolates a polynomial $\delta(X)$ from $\overline{b}$
	\end{enumerate}
	\item $\vrf$ sends $\delta(X)$ to $\prv$
	\item $\prv$ takes as input the selector vector of keys its shown ownership of, $\phi(X)$, from \bootstrap 
	\item $\prv$ constructs an accumulative polynomial $\sigma(X)$ such that:
		\begin{enumerate}
	\item $\sigma(X)-\sigma(X\omega)=\delta(X)\cdot\phi(X),X\ne{\omega^{n-1}}$
	\item $\sigma(\omega^{n-1})=\delta(\omega^{n-1})\cdot\phi(\omega^{n-1})$
		\end{enumerate}
	\item $\prv$ runs $\open(\sigma,\delta,\phi)$ and opens $\sigma(\omega^0)$ through \textbf{openToCommit}
	\item $\vrf$ accepts the proof in and only if
	\begin{enumerate}
		\item $[\sigma(X)-\sigma(X\omega)-\delta(X)\cdot{\phi(X)}]\cdot(X-\omega^{n-1})=0$
		\item $[\sigma(X)-\delta(X)\cdot{\phi(X)}]\cdot\frac{X^n-1}{X-\omega^{n-1}}=0$
	\end{enumerate}
\end{enumerate}	

% = = = 

\normalsize	
\end{framed}
\caption{The \poa proof demonstrates that the balances associated with each key in the anonymity set are included, the subset not owned by the exchange (per selector vector from \bootstrap) are zero-ed out, and remaining balances are totalled correctly in $\sigma(\omega^0)$. \label{alg:poa}}
\end{Protocol*}


The \bootstrap protocol proves that $\cm_\phi$ is a commitment to a selector polynomial $\phi(X)$ (in \bls) which marks the public keys owned by the exchange. At a given time (block number), the balances of every public key included in the anonymity set will be encoded in polynomial $\delta(X)$. The product of $\phi(X)\cdot\delta(X)$ will preserve the balance values owned by the exchange and zero-out the balance values not claimed by the exchange. The final step is produce a summation over the values in $\phi(X)\cdot\delta(X)$. The exchange will put $\phi(X)\cdot\delta(X)$ in accumulator form $\sigma(X)$ and prove its correctness. In this from, the total assets will sit at the head (first index) of $\sigma(X)$, which is $\sigma(\omega^0)$. The full details are provided in Protocol~\ref{alg:poa}.

% = = = = = = = = = = = = = = = = = = = = = = = = = = = = = = = = = = =

\section{Proof of Liabilities}

\subsection{The \pol Argument}
% !TEX root = ../../main.tex

% = = = 

\begin{Protocol*}[t!]
\begin{framed}
\footnotesize

% = = = 

$\prv$'s private input: user balances, $\{\bal_1,\bal_2,\dots,\bal_\mu\}$
\begin{enumerate}
        \item $\prv$ computes the binary decomposition (from most significant bit to least significant bit) of each balance, $\{z_j^{(\bal_i)}\}_{i\in[\mu],j\in[k]}$, such that $z_j^{(\bal_i)}\in\{0,1\}$ and $\bal_i=\sum_{j=k}^{0}2^j\cdot{z_j^{(\bal_i)}}$
%        \item Flat map the bits of each balance into a matrix
%        \[\begin{bmatrix}
%            z_1^{(\bal_1)} & z_2^{(\bal_1)} & z_3^{(\bal_1)} & \dots & z_m^{(\bal_1)} \\[3pt]
%            z_1^{(\bal_2)} & z_2^{(\bal_2)} & z_3^{(\bal_2)} & \dots & z_m^{(\bal_2)} \\[3pt]
%            z_1^{(\bal_3)} & z_2^{(\bal_3)} & z_3^{(\bal_3)} & \dots & z_m^{(\bal_3)} \\[3pt]
%            \vdots & \vdots & \vdots & \vdots & \vdots \\[3pt]
%            z_1^{(\bal_k)} & z_2^{(\bal_k)} & z_3^{(\bal_k)} & \dots & z_m^{(\bal_k)}
%        \end{bmatrix}\]
        \item $\prv$ puts the bits into accumulator form where $\chi_k^{(\bal_i)}=z_k^{(\bal_i)}$ and $\chi_i^{(\bal_i)}=2\chi_{i+1}^{(\bal_i)}+z_i^{(\bal_i)}$.  (Remark: visualized as a matrix, each row is a balance where the $k$-th column is the least significant bit and, moving right-to-left, each bit is folded in until it accumulates to $\bal_j$ in the first column.)
        \[\begin{bmatrix}
            \chi_1^{(\bal_1)} & \chi_2^{(\bal_1)} & \chi_3^{(\bal_1)} & \dots & \chi_k^{(\bal_1)} \\[3pt]
            \chi_1^{(\bal_2)} & \chi_2^{(\bal_2)} & \chi_3^{(\bal_2)} & \dots & \chi_k^{(\bal_2)} \\[3pt]
            \chi_1^{(\bal_3)} & \chi_2^{(\bal_3)} & \chi_3^{(\bal_3)} & \dots & \chi_k^{(\bal_3)} \\[3pt]
            \vdots & \vdots & \vdots & \vdots & \vdots \\[3pt]
            \chi_1^{(\bal_\mu)} & \chi_2^{(\bal_\mu)} & \chi_3^{(\bal_\mu)} & \dots & \chi_k^{(\bal_\mu)}
        \end{bmatrix}\]
        
        \item Due to the concrete parameters of \bls, $\prv$ will work column-by-column (proof size and verifier time will be linear in $k$ which is the bit-precision of each account). Let column $j$ be vector $\overline{p}_j=\{\chi_j^{(\bal_1)},\chi_j^{(\bal_2)},\dots,\chi_j^{(\bal_\mu)}\}$. The following constraints apply (for $i\in[\mu],j\in[k]$):       $\overline{p}_1[i]=\bal_i$; $\overline{p}_j[i] - 2\cdot\overline{p}_{j+1}[i]\in\{0,1\}$; and $\overline{p}_k[i]\in\{0,1\}$. $\overline{p}_1$ contains $\{\bal_1,\bal_2,\dots,\bal_\mu\}$.
        
        
        \item  $\prv$ builds an additive accumulator $\nu$ for $\overline{p}_1$ where $\nu_k=\bal_k=\overline{p}_1[k]$ and $\nu_i=\nu_{i+1}+\bal_i,i\in[1,\mu]$. Remark: $\nu_1$ will contain the total liability balances. 
        
	 \item $\prv$ interpolates polynomials for $\overline{p}_j \rightarrow p_j(X)$ and publishes commitments to each using KZG in extended zero-knowledge form (see Section~\ref{sec:kgzzkp}).
 
	
        \item $\prv$ shows the following polynomials are vanishing for all $x\in{H}$ where $H=\{\omega^0,\omega^1,\dots,\omega^{k-1}\}$
        \begin{align*}
            w_1:=&[p_0(X)-p_0(X\omega)-p_1(X)]\cdot(X-\omega^{\mu-1}) \\
            w_2:=&[p_0(X)-p_1(X)]\cdot\frac{X^\mu-1}{X-\omega^{\mu-1}} \\
            w_3:=&p_m(X)\cdot[1-p_m(X)] \\
            v_1:=&[p_1(X)-2p_2(X)]\cdot[1-(p_1(X)-2p_2(X))] \\
            v_2:=&[p_2(X)-2p_3(X)]\cdot[1-(p_2(X)-2p_3(X))] \\
            \vdots \\
            v_{k-1}:=&[p_{k-1}(X)-2p_k(X)]\cdot[1-(p_{k-1}(X)-2p_k(X))]
        \end{align*}
        $w_1$ and $w_2$ prove the accumulative vector is correct, $w_3$ and $\{v_i\}$ prove each liability is greater than or equal to 0 (range proof). To complete the proof, $\prv$ will run $\openzk(p_0,p_1,p_2,\dots,p_k)$ at a random evaluation point $\zeta$ and $\openzk(p_0)$ at $\zeta\omega$
        % This step involves the verifier sampling a random point as the challenge and the prover responding with the evaluations and the alleged commitments of each polynomial $p_i$ at that point.
        % \begin{enumerate}
        %     \item $\prv$ generates two random numbers $\omega^{\prime},\omega^{\prime\prime}$ in $\mathbb{F}\setminus{H}$ and another two random numbers $\alpha,\beta$ in $\mathbb{F}$ for each polynomial $p_i$
        %     \item $\prv$ interpolates $\{p_i\}$ at two more points ${\omega_i^{\prime},\omega_i^{\prime\prime}}$ where
        %     \[ p_i(\omega_i^{\prime})=\alpha_i \]
        %     \[ p_i(\omega_i^{\prime\prime})=\beta_i \]
        %     Note this should be done at the beginning of the protocol
        %     \item $\prv$ sends all commitments of ${p_i}$
        %     \[ \{\cm_{p_0},\cm_{p_1},\dots,\cm_{p_m}\} \]
        %     \item $\vrf$ sends a random evaluation point $\tau\in\mathbb{F}\setminus{H}$
        %     \item $\prv$ sends the evaluations $\{y_i\}$ of $\{p_i(\tau)\}$ and $p_0(\tau\omega)$
        %     \item $\vrf$ sends a random challenge $\gamma\in\mathbb{F}$
        %     \item $\prv$ computes
        %     \[ w:=w_1+\gamma w_2+\gamma^2 w_3+\sum_{i=1}^{m-1}\gamma^{i+2}v_i \]
        %     and sends the commitment to the quotient polynomial $[q_w]_1$ where
        %     \[ q_w:=w/(X^k-1) \]
        %     \item Let $\hat{w}:=w-q_w\cdot(X^k-1)$. Note $\hat{w}$ is a zero polynomial for all $x\in\mathbb{F}$, which means $X-\tau$ divides $\hat{w}$. Then $\prv$ sends $[q_{\hat{w}}]_1$ where
        %     \[ q_{\hat{w}}:=\hat{w}/(X-\tau) \]
        %     \item $\vrf$ accepts the proof if and only if $e(F,[1]_2)=e([q_{\hat{w}}]_1,[x-\tau]_2)$, where
        %     \begin{align*}
        %         F:=&(\cm_{p_0}-[p_0(\tau\omega)+p_1(\tau)]_1)\cdot(\tau-\omega^{k-1}) \\
        %          & + \gamma\cdot(\cm_{p_0}-[p_1(\tau)]_1)\cdot{\frac{\tau^k-1}{\tau-\omega^{k-1}}} \\
        %          & + \gamma^2\cdot{\cm_{p_m}-[p_m(\tau)]_1} \\
        %          & + \sum_{i\in[1,m-1]}\gamma^{i+2}\cdot[(\cm_{p_i}-[2p_{i+1}(\tau)]_1)\cdot(\cm_{p_i}-[2p_{i+1}(\tau)+1]_1)] \\
        %          & - [q_w]_1\cdot(\tau^n-1)
        %     \end{align*}
        % \end{enumerate}
    %\item $\prv$ reveals the commitments to $\{p_i\}$ from each group and the committed liablity $\pb(\liab)$
\end{enumerate}

$\vrf$ protocol:
\begin{enumerate}
    \item Verify each evaluation is valid
    \item Verify $\{w_1,w_2,w_3\}$ and $\{v_i\}$ are vanishing in $H$
\end{enumerate}




% = = = 

\normalsize	
\end{framed}
\caption{The \pol proof demonstrates that each liabilities is either zero or a positive number, and that the balances are totalled correctly in $\pi_1(\omega^0)$. \label{alg:pol}}
\end{Protocol*}


The exchange will commit to every user balance and produce a commitment of the total amount across all balances. Since the exchange is free to make-up additional users and include them, we want to make sure this does not help an insolvent exchange in any way. To do this, we force all balances to be zero or positive numbers. For a finite field, this means small integers that have no chance of exceeding the group order (modular reduction) when added together. In practice, we can limit ourselves to an even smaller range that is sufficient to capture what a balance in ETH (or fractions of ETH) might look like. These balances are expressed in binary and we use range proof from Section~\ref{sec:range}.

However when we turn to implement this in practice, we encounter a roadblock. If $\mu$ balances are placed as $k$-bit numbers side-by-side in a vector, we need a vector of size $\mu \cdot k$. If we want to optimize polynomial interpolation, we encode our array at x-coordinates that correspond to the roots of unity of the exponent group (see Appendix~\ref{app:rou}) and for \bls, we can only efficiently encode data vectors of length up to \(2^{32}=4,294,967,296\).\footnote{The exponent group in \bls has 2-adicity of 32.} Consider an exchange with $\mu=1,000,000$ accounts, only 12 bits are left to capture account balances, say, as between 0.01 and 40.96 ETH (\$30 to \$150K USD at time of writing). Exchanges could have more than 1 million accounts, the largest could be more than \$150K USD, and an exchange could have a long tale of accounts with balances less than \$30 such that rounding them all up to \$30 creates a solvency issue. Clearly \(k=2^{32}\) is not large enough for directly encoding liabilities (as binary numbers) into a single polynomial.

To deal with this issue, there are three main alternatives. (1) The exchange can encode points at arbitrary x-coordinates and use general (Laplacian) interpolation, (2) the exchange can break down what it is proving into chunks but this will require one succinct proof per chunk, or (3) the range proof could be adapted for decomposition into something larger than bits (\eg bytes or 32-bit words). The latter may be feasible with lookup arguments, but we do not pursue modifying the range proof~\cite{rangeproof} in this work. Instead we opt for (2). Specifically we will produce a polynomial argument for the first bit of every account, for the second bit of every account, \etc This means \pol will be linear in proof size and verifier work but it is linear in the bit-precision of each account ($k$) and is in fact constant (succinct) in terms of the number of users on the exchange. For example, we will later show if accounts are captured with a precision of 32-bits, the proof size will be under 10KB and verifier time will be under 8ms independent of the number of users on the exchange (see Figure~\ref{fig:pol}).  

The protocol creates $k$ polynomials---the first polynomial $p_1$ for the first bit of each of the $\mu$ accounts, the second polynomial for the second bit of every account, \etc It conducts a range proof `vertically' (across $\{p_1(i), p_2(i), \ldots, p_k(i)\}$ for $\bal_i$) for each account (for all $i$). It then converts the bits into integers `vertically' (across $p_0(i)=\sum_{j=0}^{k-1} 2^j \cdot p_j(i)$)  for each account, creating a polynomial $p_0$ of each user's balance. Last it sums up all elements `horizontally' ($\sum_{i=0}^{\mu-1} p_0(\omega^i)$) in $p_0$ to produce the total liabilities. The bit decomposition is argued with the range proof and the summation of balances is argued with a sum-check. The full protocol is given in Protocol~\ref{alg:pol}. 

% = = = = = = = = = = = = = = = = = = = = = = = = = = = = = = = = = = =

\subsection{The \userproof Argument}
% !TEX root = ../../main.tex

% = = = 

\begin{Protocol*}[t!]
\begin{framed}
\footnotesize

% = = = 

$\prv$ private input: user identifiers, $\{\uid_1,\uid_2,\dots,\uid_\mu\}$, and user balances, $\{\bal_1,\bal_2,\dots,\bal_\mu\}$
\begin{enumerate}
    \item $\prv$ interpolates the identifier polynomial $u(X)$ such that $u(X_i)=\uid_i$ for $i$ from 1 to $\mu$.
    \item $\prv$ publishes commitment to $u(X)$.
    \item For check from user $i$, $\prv$ tells the user they at index $i$ and opens $u(i)$ and $p_1(i)$ (from \pol above).
\end{enumerate}

\textbf{Customer's input:} the commitments to the identifier polynomial $u(X)$ and the balance polynomial $p_1(X)$ and the witnesses at the corresponding points of $u(i)$ and $p_1(i)$
\begin{enumerate}
    \item $\vrf$ verifies that the two indexes $i$ are identical.
    \item $\vrf$ verifies their user identifier is the evaluation of $u$ at the given point $i$.
    \item $\vrf$ verifies their balance is the evaluation of $p_1$ at the given point $i$.
\end{enumerate}



% = = = 

\normalsize	
\end{framed}
\caption{The \userproof proof demonstrates that to each user who checks that their balance is recorded correctly under a unique identifier for them (to mitigate clash attacks).\label{alg:users}}
\end{Protocol*}

The \userproof argument is conducted between the exchange and the user, so the user can check that their balance is correctly encoded into the polynomials used in \pol. If two users have the same balance, a malicious exchange might include only one of the balances and open up the same balance for each user. Unless if the users compared their proofs, they would not catch the exchange (\cf~\cite{broken}). This attack appears in other cryptographic protocols where users need to check things, the main one being cryptographic voting schemes. It has been studied under general definitions as a `clash attack'~\cite{clash}. The solution is to label each balance with a unique user identifier~\cite{provisions}. Labeling can be done with an additional polynomial of labels under the assumption that a user id and a balance needs to be at the same index. A user id can be the hash of the user's account name or email address.  The full protocol is given in Protocol~\ref{alg:users}.

% = = = = = = = = = = = = = = = = = = = = = = = = = = = = = = = = = = =

\subsection{The \pos Argument}
% !TEX root = ../../main.tex

% = = = 

\begin{Protocol*}[t!]
\begin{framed}
\footnotesize

% = = = 

\begin{enumerate}
    \item $\prv$ computes equality $\mathsf{eq}$ as the total assets minus the total liabilities. 
    \item $\prv$ publishes commitment to polynomial $\phi(X)$ where $\psi(\omega^0)=\mathsf{eq}$.
    \item $\prv$ generates a range proof for $\mathsf{eq}$ in $\phi(X)$ to demonstrate it is a non-negative integer.
    \item $\prv$ opens $\sigma(\omega^0) - p_1(\omega^0) - \psi(\omega^0)$ to the value 0.
\end{enumerate}

% = = = 

\normalsize	
\end{framed}
\caption{The \pos proof demonstrates that the total assets exceed the total liabilities by a non-negative integer (called the equity). \label{alg:pos}}
\end{Protocol*}

The final step of the proof is prove the total assets exceed the total liabilities. At the end of \poa, total assets contained in the polynomial evaluation point $\sigma(\omega^0)$; while at the end of \pol, the total liabilities are contained in $p_1(\omega^0)$. Assuming assets exceed liabilities by some amount, this amount can be added to the liability-side to provide a difference of exactly zero. The full argument is given in Protocol~\ref{alg:pos}.


% = = = = = = = = = = = = = = = = = = = = = = = = = = = = = = = = = = =
