\chapter{Security Analysis}
\label{sec:proof}

We adapt the security definition of a zero-knowledge proof of solvency from Provisions~\cite{provisions}. In Appendix~\ref{app:defs}, we recount the definitions and offer a proof sketch for the the main theorem:

\begin{restatable}{theorem}{thmmaster}
\label{thm:master}
\Sys ($\pi_\mathsf{solvency}\leftarrow\langle \pi_\mathsf{keys}, \pi_\mathsf{liabilies}, \pi_\mathsf{assets}, \pi_\mathsf{users} \rangle$) is a privacy-preserving proof of solvency with respect to Definition~\ref{def:2}. 
\end{restatable}

% = = = = = = = = = = = = = = = = = = = = = = = = = = = = = = = = = = =

\subsection{Claims}
\label{sec:claims}


\begin{restatable}{theorem}{sigmaclaim}
\label{thm:sigmaclaim}
A $\Sigma$-protocol for relation $\mathrm{ZKPoK} \{ (sk_i,s_i) :  [\mathsf{pk}_i=g^{\mathsf{sk}_i} \land p_i=\Comm{1,r_i}  ] \lor p_i=\Comm{0,r_i} \}$ exists which is complete, has special soundness, and is honest verifier zero-knowledge (HVZK).
\end{restatable}

\begin{proof}

To demonstrate completeness, consult Protocol~\ref{alg:boot} (the inner framed protocol). \\
To demonstrate special soundness, let two accepting conversations between $\prv$ and $\vrf$
$$
(t_1,t_2,t_3,e,e_1,e_2,z_1,z_2,z_3),(t_1,t_2,t_3,e^\prime,e_1^\prime,e_2^\prime,z_1^\prime,z_2^\prime,z_3^\prime)\text{ with $e\ne{e^\prime}$}
$$
be given. It is obivious we can compute $sk_i,s_i$ such that the relation $\mathrm{ZKPoK}(sk_i,s_i)$ exists. Thus the $\Sigma$-protocol for the relation $\mathrm{ZKPoK}$ has special soundness. \\
Honest verifier zero-knowledge (HVZK) is clear: given $e,z_1,z_2,z_3$ at random and choose $e_1,e_2$ such that $e=e_1\oplus{e_2}$, we can make a simulated conversation between the honest verifier and prover using the faking proof tricks. Since $e,z_1,z_2,z_3$ can be chosen freely, the simuated conversation is not distinguishable from the real one.

Since the probability that $s$ is equal to zero or one is exactly the same as the real $\prv$ does, $\vrf$ cannot distinguish the proof produced by $\mathcal{S}$ from the one generated by $\prv$. \end{proof}


% = = = = = = = = = = = = = = = = = = = = = = = = = = = = = = = = = = =

\begin{restatable}{theorem}{nizkclaim}
\label{thm:nizkclaim}
A $\Sigma$-protocol for relation $\mathrm{ZKPoK} \{ (sk_i,s_i) :  [\mathsf{pk}_i=g^{\mathsf{sk}_i} \land p_i=\Comm{1,r_i}  ] \lor p_i=\Comm{0,r_i} \}$ exists which is a non-interactive zero knowledge proof (NIZKP).
\end{restatable}

\begin{proof}
Given the relation can proven with a ``standard'' $\Sigma$-protocol (per Thereom~\ref{thm:sigmaclaim}), we can use the well-known Fiat-Shamir heuristic to compile it to a NIZKP in the random oracle model. We do not repeat the proof for this (see~\cite{damgard10,sigma}) but stress that strong Fiat-Shamir~\cite{weakfs} needs to be used here and in the Poly-IOP components of \Sys, or practical attacks could be leveraged against the system (\cf~\cite{weakfsattacks}).\end{proof}

% = = = = = = = = = = = = = = = = = = = = = = = = = = = = = = = = = = =