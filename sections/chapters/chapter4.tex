\chapter{Security Analysis}
\label{sec:proof}

We adapt the security definition of a zero-knowledge proof of solvency from Provisions~\cite{provisions} in Appendix~\ref{app:defs}. We offer security proofs for \Sys and its sub-components: \bootstrap, \userproof, \pol, \poa, and \pos.

\section{Claims}

% = = = = = = = = = = = = = = = = = = = = = = = = = = = = = = = = = = =

\begin{theorem}
\label{thm:sigmaclaim}
A $\Sigma$-protocol for relation $\mathrm{ZKPoK}\{(sk_i,s_i):[\mathsf{pk}_i=g^{\mathsf{sk}_i}\land{p_i}=\Comm{1,r_i}] \lor{p_i}=\Comm{0,r_i}\}$ exists which is complete, has special soundness and is special HVZK. 
\end{theorem}

\begin{proof}
To demonstrate completeness, consult Protocol~\ref{alg:boot} (the inner framed protocol).

To demonstrate special soundness, let two accepting conversations between $\prv$ and $\vrf$
$$
(t_1,t_2,t_3,e,e_0,e_1,z_1,z_2,z_3),(t_1,t_2,t_3,e^\prime,e_0^\prime,e_1^\prime,z_1^\prime,z_2^\prime,z_3^\prime)\text{ with $e\ne{e^\prime}$}
$$
be given. It is obivious for some $s=0$ or $1$ and $e_s\ne{e_s^\prime}$, we can compute $sk_i,s_i$ from the above conversations. Thus, the $\Sigma$-protocol for the relation $\mathrm{ZKPoK}$ has special soundness.

To demonstrate special HVZK, given $e$ and randomly choose $e_0,e_1$ such that $e=e_1\oplus{e_2}$, let the simulator compute
$$
z_1\stackrel{\$}{\leftarrow}\mathbb{Z}_s,z_2\stackrel{\$}{\leftarrow}\mathbb{Z}_b,z_3\stackrel{\$}{\leftarrow}\mathbb{Z}_b,t_1=\gs^{z_1}/\mathsf{pk}_i^{e_0},t_2=\gb^{e_0}\hb^{z_2}/{p_i^{e_0}},t_3=\hb^{z_3}/{p_i^{e_1}}
$$
and output $(t_1,t_2,t_3,e,e_0,e_1,z_1,z_2,z_3)$. Clearly, the transcript is accepted by $\vrf$. Note $e$, $e_0$ and $e_1$ are random $t$-bit strings, which means they have the same distribution as the conversation between $\prv$ and $\vrf$. $z_1$, $z_2$, and $z_3$ are uniformly distributed over their corresponding fields; moreover, given $(e_1,e_2,z_1,z_2,z_3)$, $(t_1,t_2,t_3)$ are uniquely determined by the above equations. Therefore, the simuated transcript is not distinguishable from the real one to $\vrf$.
\end{proof}

% = = = = = = = = = = = = = = = = = = = = = = = = = = = = = = = = = = =

\begin{restatable}{theorem}{nizkclaim}
\label{thm:nizkclaim}
A $\Sigma$-protocol for relation $\mathrm{ZKPoK} \{ (sk_i,s_i) :  [\mathsf{pk}_i=g^{\mathsf{sk}_i} \land p_i=\Comm{1,r_i}  ] \lor p_i=\Comm{0,r_i} \}$ exists which is a non-interactive zero knowledge proof (NIZKP).
\end{restatable}
    
\begin{proof}
Given the relation can proven with a ``standard'' $\Sigma$-protocol (Thereom~\ref{thm:sigmaclaim}), we can use the well-known Fiat-Shamir heuristic to compile it to a NIZKP in the random oracle model. We do not repeat the proof for this (see~\cite{damgard10,sigma}) but stress that strong Fiat-Shamir~\cite{weakfs} needs to be used here and in the Poly-IOP components of \Sys, or practical attacks could be leveraged against the system (\cf~\cite{weakfsattacks}).
\end{proof}

% = = = = = = = = = = = = = = = = = = = = = = = = = = = = = = = = = = =

\section{The \bootstrap Argument}

\begin{claim}
\label{thm:keys}
\bootstrap is a complete, sound, and NIZKP.
\end{claim}

\begin{proof}
Recall that the \bootstrap argument contains the relation proven to be NIZKP in Theorem~\ref{thm:nizkclaim}. For the outer frame in Protocol~\ref{alg:boot}, $\vrf$ does not need to send any challenge to $\prv$, and $\prv$ instead opens the evaluations of the selector polynomial at each root of unity, which mean \bootstrap is non-interactive. It remains to be shown the rest of the protocol (outer frame in Protocol~\ref{alg:boot}) is secure.

Completeness follows from Protocol~\ref{alg:boot}. The remainder of the protocol involves $\prv$ demonstrating that the selector polynomial encodes a 1 at index $\omega^i$ if and only if the corresponding $i$-th run of the $\Sigma$-protocol used $s=1$, and contains a 0 otherwise.

For \bootstrap to be sound, it requires (i) the polynomial commitment scheme (PCS) to be binding and (ii) the PCS to have a sound point-evaluation argument. These two properties are both demonstrated for KZG in the original paper~\cite{kzg}. Specifically these two properties rely on four assumptions:
\begin{itemize}
\item \textbf{KZG.A1}: The trusted setup outputs a structured reference string (SRS) with the value of $\tau$ unknown to $\prv$. 
\item \textbf{KZG.A2}: The value of $\tau$ cannot be extracted from the SRS which assumes $\prv$ is computationally bounded and relies on (for us in \bls) the $t$-strong Diffie-Hellman (t-SDH) assumption.
\item \textbf{KZG.A3}: If an adversary interpolates a polynomial through the point $(\omega^i,y)$ such that $y=P(\omega^i)$ but claims $y'=P(\omega^i)$ for some $y'\neq y$ then the probability that $\tau-y'$ evenly divides $y=P(\tau)$ is overwhelmingly low. This property can be demonstrated using the Schwartz-Zippel lemma by showing the number of $\tau$ values satisfying this property is bounded from above by $d$/$q$ where $d$ is the degree of the polynomial and $q$ is the size of the exponent group. For \bls with 255-bit exponents and 2-adicity of 32, this is close to $2^{32-255}=2^{-233}$ which is negligible.
\end{itemize}

Finally, in our protocol $\prv$ does not reveal the evaluation of the polynomial at a point, $\prv$ instead reveals a commitment to the evaluation through \textbf{openToCommit}. In the original KZG opening scheme, $\prv$ opens the commitment to the polynomial first and then opens the evaluation at the challenge point. Our modification just moves the computation work for the committed evaluation from $\vrf$ to $\prv$. For the security proof of \textbf{openToCommit}, see Theorem~\ref{them:kzgOpen}. Therefore, \bootstrap has zero knowledge.

For future claims, we encapsulate all assumptions about KGZ as a \textit{polynomial oracle}. 
\end{proof}

% = = = = = = = = = = = = = = = = = = = = = = = = = = = = = = = = = = =
